\documentclass[fleqn,12pt, a4paper, oneside]{article}
\usepackage[%
    left=1.5cm,%
    right=1.5cm,%
    top=2.0cm,%
    bottom=2.0cm,%
    % paperheight=11in,%
    % paperwidth=8.5in%
]{geometry}%
\usepackage{refcount}% http://ctan.org/pkg/refcount
\usepackage{tikz,pgfplots}
\usepackage{mathtools}
\usepackage{xcolor}
\usepackage{amsmath}
\usepackage{hyperref}
\usepackage{xepersian}
\usetikzlibrary{3d}
\usetikzlibrary{calc}
% declaring \sech and \csch functions
\DeclareMathOperator{\sech}{sech}
\DeclareMathOperator{\csch}{csch}
\DeclareMathOperator{\arccot}{arccot}
\settextfont{Vazirmatn}
\begin{document}
\tableofcontents
\newpage
\section{مشتق}
به آهنگ تغییرات تابع، یعنی شیب خط مماس بر یک تابع، مشتق آن تابع میگویند. که به صورت 
\subsection{تعریف مشتق}
    \[ f'(x) = \lim_{h\to 0} \frac{f(x)-f(x+h)}{h} \]
    \[ f'(a) = \lim_{x\to a} \frac{f(x)-f(a)}{x-a} \]

\subsection{قوانین مشتق}
    \[u,v = f(x); a \in \mathbb{R} \]
    \[ (u\pm v)'=u'\pm v' \]
    \[ (au)'=au'\]
    \[(u \cdot v)'=u'\cdot v+v' \cdot u\]
    \[(\frac{ u }{ v })'=\frac{u'\cdot v-v' \cdot u}{v^2}\]
\subsection{فرمول های مشتق}
    \subsubsection{فرمول های جبری مشتق}
    \[(u^n)'=nu'u^{n-1}\]
    \[|u|' = \frac{u'\cdot u}{|u|}\]
    \[(\frac{ au+b }{ cu+d })'=\frac{ad-bc}{(cu+d)^2}u'\]
    \[(a^u)'=u'a^u\ln{a}\]
    \[(\ln u)' = \frac{u'}{u}\]
    \subsubsection{فرمول های مثلثاتی مشتق}
    \[ \sin{u}' = u'\cos{u}\]
    \[ \cos{u}' = -u'\sin{u}\]
    \[ \tan{u}' = u'(1+\cot^2{u})\]
    \[ \cot{u}' = -u'(1+\tan^2{u})\]
    \[ \sec{u}' = u'\sin{u}\cdot\sec^2{u}\]
    \[ \csc{u}' = u'\cos{u}\cdot\csc^2{u}\]
    \[\arcsin{u}' = \frac{u'}{\sqrt{1-u^2}}\]
    \[\arccos{u}' = \frac{-u'}{\sqrt{1-u^2}}\]
    \[\arctan{u}' = \frac{u'}{1+u^2}\]
    \[\arccot{u}' = \frac{-u'}{1+u^2}\]
    \subsubsection{فرمول های هذلولی مشتق}
    فرمول های هذلولی،‌ مثل فرمول های مثلثاتی هستند، با تفاوت های زیر: (سینوس تفاوتی ندارد)
    \[ \sinh{u}' = u'\cosh{u}\]
    \[ \cosh{u}' = u'\sinh{u}\]
    \[ \tanh{u}' = u'(1-\coth^2{u})\]
    \[ \coth{u}' = u'(1-\tanh^2{u})\]
\end{document}