\documentclass[12pt, a4paper, oneside]{article}
\usepackage[%
    left=1.5cm,%
    right=1.5cm,%
    top=2.0cm,%
    bottom=2.0cm,%
    % paperheight=11in,%
    % paperwidth=8.5in%
]{geometry}%
\usepackage{refcount}% http://ctan.org/pkg/refcount
\usepackage{tikz,pgfplots}
\usepackage{mathtools}
\usepackage{hyperref}
\usepackage{xepersian}
\usetikzlibrary{3d}
\usetikzlibrary{calc}

% tikz and pgfplot for axis
\pgfplotsset{every axis/.append style={
                    axis x line=middle,    % put the x axis in the middle
                    axis y line=middle,    % put the y axis in the middle
                    axis line style={<->}, % arrows on the axis
                    xlabel={$x$},          % default put x on x-axis
                    ylabel={$y$},          % default put y on y-axis
                    }}

% arrows as stealth fighters
\tikzset{>=stealth}

\settextfont{Vazirmatn}

% uses "refcount" package. for counting footnotes
\newcounter{fnotecounter}
\begin{document}
%  commands and variables:
\newcommand\deltavec[1]{\vec{\Delta{#1}}}
\newcommand\undersetrtl[2]{${\underset{#1}{#2}}$}
\newcommand\fnote[1]{\footnote[\refstepcounter{fnote}]}
\section{یادآوری های دبیرستان}
\subsection{لگاریتم}
\label{subsec:log}
لگاریتم، یک عدد در یک پایه، برابر با توانی از پایه است که آن عدد را می دهد.
\[a^b = c \Leftrightarrow \log_a (c) = b\]
برای مثال: 
\[10^3 = 1000 \Leftrightarrow \log_{10} (1000) = 3\]
در سیستم اعداد دسیمال
\footnote[1]{
    در سیستم عدد باینری، 
    $\log{n}$
    به معنی لگاریتم n به پایه ی ۲ است.
    مثل 
    bigO ی 
    الگوریتم های
    divide and conquer
}
    ، پایه ی لگاریتم ۱۰ را نمینویسیم. مثال بالا معمولا به صورت زیر نوشته میشود: 
    \[\log (1000) = 3\]
\subsubsection{قوانین لگاریتم}

\noindent
\begin{latin}
\begin{itemize}
    \item { $\log_a(x) = \frac{ 1 }{ \log_x(a) }$ }
    \item { $\log_a(x) = \log_{10}(x)/\log_{10}(a)$ }
    \item { $\log_a(xy) = \log_a (x) + \log_a(y)$ }
    \item { $\log_a(\frac{x}{y}) = \log_a (x) - \log_a(y)$ }
    \item { $\log_a(x^y) = y.\log_a(x)$ }
    \item { $\log_a (m) = \log_a(n) \Leftrightarrow m = n$ }
\end{itemize}
\end{latin}

\subsection{مثلثات}
\subsubsection{نسبت های مثلثاتی}
\begin{latin}
\begin{itemize}
    \item $\sin^2x + \cos^2x = 1$
    \item $\sin(x \pm y) = \sin(x)\cos(y) \pm \cos(x)\sin(y)$
    \item $\cos(x \pm y) = \cos(x)\cos(y) \mp \sin(x)\sin(y)$
    \item $\sec x= \frac{1}{\cos x}$
    \item $\csc x= \frac{1}{\sin x}$
    \item $1+\tan^2x = \sec^2x$
    \item $1+\cot^2x = \csc^2x$
\end{itemize}
\end{latin}

\section{درس های جدید}
\subsection{اعداد مختلط}
اعداد مختلط اعدادی هستند که به صورت روبرو نمایش در می آیند: $Z = x + yi$
که به \undersetrtl{}{x} مولفه حقیقی و به \undersetrtl{}{y} مولفه موهومی گفته میشود.

\undersetrtl{}{y} مولفه ی موهومی است، چون ضریب عدد موهومی \footnote[2]{عدد موهومی(imaginary): $i=\sqrt[]{-1}$} است.
\subsection{عدد نپر (اویلر)}
عدد نپر به صورت زیر تعریف میشود:
\[e=\sum_{n=0}^{\infty} \frac{1}{n!}\]
لگاریتمی
{$\overset{\ref{subsec:log}}{}$}
 که مبنای آن عدد نپر باشد را به صورت زیر تعریف میکنیم: 
\[\log_e (x) = \ln (x)\]
\subsection{توابع هایپربولیک}
توابع هایپربولیک توابعی هستند که بجای دایره ی مثلثاتی به نسبت دو هذلولی واحد متانسب میشوند.
\begin{tikzpicture}
    \begin{axis}[
            xmin=-3,xmax=3,
        ymin=-3,ymax=3]
        \addplot [blue,thick,domain=-1.5:1.5] ({cosh(x)}, {sinh(x)});
        \addplot [blue,thick,domain=-1.5:1.5] ({-cosh(x)}, {sinh(x)});
    \end{axis}
\end{tikzpicture}
\end{document}